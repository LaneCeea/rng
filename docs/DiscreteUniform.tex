\documentclass[12pt]{article}
\usepackage[a4paper, total={17.18cm, 24.62cm}]{geometry}
\usepackage[onehalfspacing]{setspace}
\usepackage{amssymb}
\usepackage{amstext}
\usepackage{amsmath}
\usepackage{mathtools}
\usepackage{listings}
\usepackage{xcolor}

\lstdefinestyle{pseudocode}{
    basicstyle=\ttfamily\normalsize,
    keywordstyle=\bfseries\color{blue},
    keywordstyle=[2]\bfseries\color{purple},
    commentstyle=\itshape\color{gray},
    stringstyle=\color{red},
    numberstyle=\small\color{gray},
    language={},
    keywords={if, else, while, for, to, downto, return, and, or, free, swap, floor, ceil, print, max},
    keywords=[2]{NIL, inf, true, false, uint},
    frame=single,
    frameround=tttt,
    framesep=5pt,
    numbers=left,
    numbersep=10pt,
    breaklines=true,
    breakatwhitespace=false,
    tabsize=4,
    showstringspaces=false,
    captionpos=b,
    backgroundcolor=\color{gray!5},
    lineskip=2pt,
    morecomment=[l]{//}
}

\lstnewenvironment{pseudocode}[1][]
{
    \lstset{style=pseudocode, #1}
}
{}

\begin{document}

\section{Problem}

\textbf{Input:} A PRNG engine that generates random integers uniformly distributed on \([0, n)\). \\
\textbf{Output:} A random integer \(y\) uniformly distributed on \([0, s)\), where \(0 < s \leq n\).

\section{Rejection Sampling}

\subsection{Algorithm}

\begin{pseudocode}
uint Uniform(Prng, uint s)
    uint q = Prng.n / s     // integer division
    uint m = q * s
    while true
        uint x = Prng()     // Unif{0, ..., n-1}
        if x < m
            return x % s
\end{pseudocode}

\subsection{Analysis}

\subsubsection{Notation}

\begin{itemize}
    \item \(q = \lfloor n / s \rfloor\) and \(m = qs\) (the largest multiple of \(s\) not exceeding \(n\))
    \item Let \(X_1, X_2, \dots\) be the \textit{i.i.d.} outputs of successive PRNG calls, with
    \[
        X_t \sim \mathrm{Unif}\{0, \dots, n-1\}.
    \]
    \item The (surely finite) stopping time
    \[
        T = \min\{t : X_t < m\}.
    \]
    \item The algorithm returns
    \[
        Y = X_T \bmod s.
    \]
\end{itemize}

\subsubsection{Average Time Complexity}

At each draw we have the acceptance event \(A_t = \{X_t < m\}\) with
\[
    \mathbb{P}(A_t) = \frac{m}{n}.
\]
Thus, \(T\) is geometric with success probability \(m/n\), so
\[
    \mathbb{P}(T < \infty) = 1 \quad \text{and} \quad \mathbb{E}[T] = \frac{n}{m}.
\]
This implies that \(T\) is finite, so the algorithm always terminate and is expected to draw \(n/m\) times from PRNG. Therefore, the average time complexity of the algorithm is
\[
    O \left( \frac{n}{m} \right) \cdot \Bigl( \text{the time complexity of } \verb|Prng()| \Bigr),
\]
and we can further claim that \(O(n/m)\) is actually \(O(1)\).

\paragraph{Proof.} Since \(m\) is the largest multiple of \(s\) not exceeding \(n\), we can write
\[
    n = m + r, \quad \text{where } 0 \leq r < s.
\]
Then
\[
    \frac{n}{m} =
    \frac{m + r}{m} =
    1 + \frac{r}{m} <
    1 + \frac{s}{m} =
    1 + \frac{1}{q} \leq 2
\]

\subsubsection{Correctness}

Let \(A = \{X_1 < m\}\) be the acceptance event of the first draw. Then for any \(y \in \{0, \dots, s-1\}\),
\[
    \mathbb{P}(Y = y) \; = \; \mathbb{P}(A) \, \mathbb{P}(y = X_1 \bmod s \mid A) + \mathbb{P}(A^c) \, \mathbb{P}(Y = y),
\]
since \(A\) is the event that we accept immediately, otherwise we ``restart'' memorylessly.

Now we have \(\mathbb{P}(A) = m/n = qs / n\) and, conditional on \(A\), \(X_1\) is uniform on \(\{0, \dots, m-1\}\), which has exactly \(q\) residues congruent to \(y \bmod s\), hence
\[
    \mathbb{P}(y = X_1 \bmod s \mid A) = \frac{q}{m} = \frac{1}{s}.
\]
Pluggin in we have
\[
    \mathbb{P}(Y = y) = \frac{qs}{n} \cdot \frac{1}{s} + \left( 1 - \frac{qs}{n} \right) \mathbb{P}(Y = y)
\]
\[
    \Rightarrow \mathbb{P}(Y = y) = \frac{1}{s}.
\]


\end{document}